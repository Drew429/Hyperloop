
%  next 2 lines pull in the NASA.cls file, put options between []
\documentclass[]             % options: RDPonly, coveronly, nocover
{aiaa-tc}                       %   plus standard article class options

%  Fill in the following catagories between the braces {}
\title{Open-Source Conceptual Sizing Models for the Hyperloop Passenger Pod}

\author{
  Jeffrey C. Chin%
     \thanks{Aerospace Engineer, MDAO Branch, Mail Stop 5-11, AIAA Member},
  \ Justin S. Gray,%
     \thanks{Aerospace Engineer, MDAO Branch, Mail Stop 5-10, AIAA Member}
  \ Scott M. Jones,%
     \thanks{Senior Systems Engineer, MDAO Branch, Mail Stop 5-11, AIAA Senior Member}
   \\
  {\normalsize\itshape
  NASA Glenn Research Center, Cleveland, OH}
}

% Define commands to assure consistent treatment throughout document
\newcommand{\eqnref}[1]{(\ref{#1})}
\newcommand{\class}[1]{\texttt{#1}}
\newcommand{\package}[1]{\texttt{#1}}
\newcommand{\file}[1]{\texttt{#1}}
\newcommand{\BibTeX}{\textsc{Bib}\TeX}

\setlength{\abovecaptionskip}{0pt}
\setlength{\belowcaptionskip}{0pt}

\usepackage{minted} %syntax highlighting
\usepackage{color} %syntax highlighting
\usepackage{graphicx}
\usepackage{changepage}
\usepackage{amsmath} %math equations
\usepackage{subfiles} %break up document into subfiles
\usepackage{hyperref} %hyperlinks
\usepackage{courier} %courier font for variable names
\usepackage{cleveref} %section references
\usepackage[all]{hypcap} %figure references point to top of figure (must come after hyperref)
\usepackage{nomencl} %nomenclature
\usepackage{subfiles} %split sections into separate docs

\usepackage{setspace} %Pro­vides sup­port for set­ting the spac­ing be­tween lines in a doc­u­ment
\usepackage{wrapfig} %Al­lows fig­ures or ta­bles to have text wrapped around them.
\usepackage{caption} %The cap­tion pack­age pro­vides many ways to cus­tomise the cap­tions in float­ing en­vi­ron­ments like fig­ure and ta­ble
\usepackage{lscape} %Mod­i­fies the mar­gins and ro­tates the page con­tents but not the page num­ber.
\usepackage{appendix}
\usepackage{listings} %Code
\usepackage[section]{placeins} %De­fines a \FloatBar­rier com­mand, be­yond which floats may not pass; 
\usepackage[superscript]{cite} %The pack­age sup­ports com­pressed, sorted lists of nu­mer­i­cal ci­ta­tions
\usepackage{esdiff} %The pack­age makes writ­ing deriva­tives very easy.
\usepackage{bm} %The bm pack­age de­fines a com­mand \bm which makes its ar­gu­ment bold. (math)
\usepackage{booktabs} % en­hances the qual­ity of ta­bles

\lstset{frame=single}
\newcommand{\txt}{\textrm}

\newcommand{\cent}{{\mathrm{c}\mkern-6.5mu{\mid}}}

\captionsetup[figure]{margin=5pt,font=small,labelfont=bf,textfont=bf,justification=justified,}
%\captionsetup[wrapfigure]{margin=5pt,font=small,labelfont=bf,justification=justified,singlelinecheck=off}
\captionsetup[table]{margin=5pt,font=small,labelfont=bf,textfont=bf,justification=justified,position=top}

\bibliographystyle{aiaa}

\usepackage{lettrine} %The let­trine pack­age sup­ports var­i­ous dropped cap­i­tals styles
\usepackage{verbatim} %reimplements verbatim, block comments

%nomenclature
\makenomenclature
%these seemed to be necessary with minted... not sure what they do..
\makeatletter
\color{black}
\let\default@color\current@color
\makeatother

%reference appendix
\crefname{appsec}{Appendix}{Appendices}
\begin{document}

\maketitle

\begin{abstract}
Hyperloop is a new mode of transportation proposed as an alternative to a high speed rail system between Los Angeles and San Francisco. It consists of a passenger pod traveling through a sealed tube under a light vacuum. The pod travels in the tube at very high speeds providing about a 30 minute travel time between Los Angeles and San Francisco. The first hyperloop design study, from a team of engineers at SpaceX and Tesla Motors, predicted that the the system would consist of a 1.1 meter tall passenger capsule traveling through an evacuated tube with a 2.2 meter diameter at 700 miles per hour. The passenger pod had an on-board compression system with a water-to-steam based heat exchanger  that reduced the aerodynamic drag as it moved through the tube. They concluded that the hyperloop could provide a substantial increase in performance over a high speed rail, while simultaneously decreasing overall costs. Although the results looked very promising, the original study was done assuming that tube size and pod size were independently selectable and without considering the effect of the compressor system on the pod size. In-fact the tube size, pod size, compression system, and heat exchangers are all tightly linked through thermodynamic effects of the air in the tube. In this work, we developed a new sizing method for the hyperloop based on 1-dimensional thermodynamic relationships that accounts for the strong coupling between these sub-systems. Using the new method we determined that the tube would need to be nearly double the size originally proposed (about 4 meters in diameter) and the maximum pod travel speed was lower than original considered (about 620 miles per hour). In addition, we found that the water-to-steam heat exchanger system was not necessary to achieve reasonable equilibrium air temperatures in the tube. Although the results indicate that hyperloop will need to be larger and slightly slower than originally proposed, the travel time increased by less than 10 minutes. So the overall performance was not dramatically affected. Furthermore the removal of the heat exchangers represents a significant simplification by removing an entire sub-system. The concept still appears viable and very attractive, although much more analysis is still to be done. 


\end{abstract}  

\setcounter{secnumdepth}{1}
\setcounter{tocdepth}{1}
%\tableofcontents
%\listoffigures
%\listoftables
\printnomenclature
 
\subfile{hyperloop}

\bibliographystyle{unsrt} %sorted by appearence, consider using natbib?
\bibliography{bibliography}

\subfile{appendix}

\end{document}