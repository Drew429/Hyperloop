
%  next 2 lines pull in the NASA.cls file, put options between []
\documentclass[]             % options: RDPonly, coveronly, nocover
{aiaa-tc}                       %   plus standard article class options

%  Fill in the following catagories between the braces {}
\title{Open-Source Conceptual Sizing Models for the Hyperloop Passenger Pod}

\author{
  Jeffrey C. Chin%
     \thanks{Aerospace Engineer, MDAO Branch, Mail Stop 5-11, AIAA Member},
  \ Justin S. Gray,%
     \thanks{Aerospace Engineer, MDAO Branch, Mail Stop 5-10, AIAA Member}
  \ Scott M. Jones,%
     \thanks{Senior Systems Engineer, MDAO Branch, Mail Stop 5-11, AIAA Senior Member}
   \\
  {\normalsize\itshape
  NASA Glenn Research Center, Cleveland, OH}
}

% Define commands to assure consistent treatment throughout document
\newcommand{\eqnref}[1]{(\ref{#1})}
\newcommand{\class}[1]{\texttt{#1}}
\newcommand{\package}[1]{\texttt{#1}}
\newcommand{\file}[1]{\texttt{#1}}
\newcommand{\BibTeX}{\textsc{Bib}\TeX}

\setlength{\abovecaptionskip}{0pt}
\setlength{\belowcaptionskip}{0pt}

\usepackage{minted} %syntax highlighting
\usepackage{color} %syntax highlighting
\usepackage{graphicx}
\usepackage{changepage}
\usepackage{amsmath} %math equations
\usepackage{subfiles} %break up document into subfiles
\usepackage{hyperref} %hyperlinks
\usepackage{courier} %courier font for variable names
\usepackage{cleveref} %section references
\usepackage[all]{hypcap} %figure references point to top of figure (must come after hyperref)
\usepackage{nomencl} %nomenclature
\usepackage{subfiles} %split sections into separate docs

\usepackage{setspace} %Pro­vides sup­port for set­ting the spac­ing be­tween lines in a doc­u­ment
\usepackage{wrapfig} %Al­lows fig­ures or ta­bles to have text wrapped around them.
\usepackage{caption} %The cap­tion pack­age pro­vides many ways to cus­tomise the cap­tions in float­ing en­vi­ron­ments like fig­ure and ta­ble
\usepackage{lscape} %Mod­i­fies the mar­gins and ro­tates the page con­tents but not the page num­ber.
\usepackage{appendix}
\usepackage{listings} %Code
\usepackage[section]{placeins} %De­fines a \FloatBar­rier com­mand, be­yond which floats may not pass; 
\usepackage[superscript]{cite} %The pack­age sup­ports com­pressed, sorted lists of nu­mer­i­cal ci­ta­tions
\usepackage{esdiff} %The pack­age makes writ­ing deriva­tives very easy.
\usepackage{bm} %The bm pack­age de­fines a com­mand \bm which makes its ar­gu­ment bold. (math)
\usepackage{booktabs} % en­hances the qual­ity of ta­bles

\lstset{frame=single}
\newcommand{\txt}{\textrm}

\newcommand{\cent}{{\mathrm{c}\mkern-6.5mu{\mid}}}

\captionsetup[figure]{margin=5pt,font=small,labelfont=bf,textfont=bf,justification=justified,}
%\captionsetup[wrapfigure]{margin=5pt,font=small,labelfont=bf,justification=justified,singlelinecheck=off}
\captionsetup[table]{margin=5pt,font=small,labelfont=bf,textfont=bf,justification=justified,position=top}

\bibliographystyle{aiaa}

\usepackage{lettrine} %The let­trine pack­age sup­ports var­i­ous dropped cap­i­tals styles
\usepackage{verbatim} %reimplements verbatim, block comments

%nomenclature
\makenomenclature
%these seemed to be necessary with minted... not sure what they do..
\makeatletter
\color{black}
\let\default@color\current@color
\makeatother

%reference appendix
\crefname{appsec}{Appendix}{Appendices}
\begin{document}

\maketitle

\begin{abstract}

In order to develop a comprehensive system model of the Hyperloop transportation system, first steps have been taken in constructing
various baseline models as outlined in Elon Musk's proposal. A completely open-source multidisciplinary framework is provided as a starting point
and the system's design, modeling, and analysis is approached from an aeronautic perspective. The modeling is
focused on the compressor cycle analysis, heat estimates, and capsule geometry. From this analysis, physical limitations of the tube diameter and
capsule geometry have been identified. These limitations are not vetted in the original proposal. 
The results indicate that travel speeds around Mach 0.8 are
attainable if a tube inner diameter approximately two times larger than originally proposed. In addition, some basic sizing estimates are made to
quantify the effect of the maximum speed on battery sizing and cooling requirements. The estimated on-board power
requirement of 300-350 kw-hrs is in agreement with Musk's original proposal. Conversely, preliminary cooling calculations indicate
 steam tanks to be an ineffective and unnecessary mechanism for managing heat loads between the tube and surrounding environment.
 \end{abstract}  

\setcounter{secnumdepth}{1}
\setcounter{tocdepth}{1}
%\tableofcontents
%\listoffigures
%\listoftables
\printnomenclature
 
\subfile{hyperloop}

\bibliographystyle{unsrt} %sorted by appearence, consider using natbib?
\bibliography{bibliography}

\subfile{appendix}

\end{document}